\documentclass{standalone}
\begin{document}
	\subsection{Training}
	
	This script allows to estimate the set of centroids. It requires as input a path to the training dataset. Each training scan must be saved in a format supported by \textsc{SimpleITK}. This script will returns as output the set of centroids in \textit{.npy} format.
	
	This script will read the training dataset,  construct the multichannel image, create the subsamples and cluster them independently. Notice that during the clustering process the background must be removed. These script is summarized in \ref{alg:training}.
	
	
		\begin{algorithm}
		
		\SetAlgoLined
		\DontPrintSemicolon
		
		\KwData{CT scans with Extracted lung}
		\KwResult{Centroid matrix}\;
		
		\ForEach{$scan \in input\_scans$}{
			
			read the scan\;
			sample$\leftarrow$image\_array\;
		}\;
		\tcc{prepare subsamples}
		sample$\leftarrow$ build\_multichannel(sample)\;
		sample$\leftarrow$shuffle(sample)\;
		subsamples$\leftarrow$split(sample)\;
		
		\tcc{start the first clustering}
		\ForEach{$ Sub \in subsamples $}
		{
			center$\leftarrow$kmeans(sub, number of centroids)\;
			centroid\_vector, internal\_variance$\leftarrow$append(center)
			
		}\;
		
		\tcc{Refinement}
		centroid\_matrix$\leftarrow$centroid\_vactor($\min(internal\_variance$))\;
		
		\caption{training}\label{alg:training}
		
	\end{algorithm}

	The clustering process is managed by the \textsc{kmeans\_on\_subsamples} fucntions, implemented as follows. 
	\lstset{style=python}
	\begin{lstlisting}[language=python, caption=kmenas\_on\_subsamples, label=code:kmeans]
		
	import cv2
	import numpy as np
	from tqdm import tqdm
		
	def kmeans_on_subsamples(imgs,  n_centroids, stopping_criteria, 
							 centr_init, weight = None) :
		
		if weight is not None :
			vector = np.asarray([ el[w != 0] 
								for el, w in zip(imgs, weight)], 
								dtype = np.ndarray)
		else :
			vector = np.asarray([el.reshape((-1, el[-1])) 
									for el in imgs], 
									dtype=np.ndarray)
		
		centroids = []
		ret = []
		for el in tqdm(vector) :
		
			r, _, centr = cv2.kmeans(el.astype(np.float32), n_centroids, 
									None, stopping_criteria, 
									10, centr_init)				
			centroids.append(centr)
			ret.append(r)
			
		final_center = centroids[np.argmin(ret)]
		
		return np.asarray(final_center, dtype= np.float32)
		
	\end{lstlisting}


	This function takes as input the array containing the subsamples, and apply kmeans clustering on each of them, storing all the values.
	To remove the background it use a weight tensor. The values of this tensor must be $0$ for the voxel corresponding to the background, and $1$ otherwise. 
	At the end of the procedure, the centroid set which provide the minum value of the sum of square error is selected.



	
	
	
\end{document}