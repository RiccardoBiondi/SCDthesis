\begin{document}
	
	\subsection{Labeling}
	
	
	
	
	This is the last step of the pipeline and involves assigning of  each voxels to the cluster corresponding to the nearest centroids, in this way an hard segmentation is achieved.
	
	The script takes as input the CT scan after the lung extraction and it build the multichannel image as described before. After that it assigns each voxel to the cluster of nearest centroids, which is the one that minimize the distance : 
	\begin{equation}
		cluster = \arg\min_{S}  \sum_{i=1}^k \sum_{S} \| x - \mu_i\|
	\end{equation}
	
	where $x$ is the color vector of the voxel and $\mu$ is the $ith$ centroid. During this process the background is automatically assigned to the 0 label, by passing a mask which assume $0$ on the volxel background and $1$ for the other one. At the end of the assignment, only the cluster corresponding to GGO and CS is selected.
	To summarize the process, the pseudocode of the script is reported in algorithm\,\ref{alg:labeling}.
	I've tested this algorithm on three different dataset, the results are described in the next chapter.
	
	\begin{algorithm}
		
		\SetAlgoLined
		\DontPrintSemicolon
		
		\SetKwFunction{Flabel}{imlabeling}
		\SetKwProg{Fn}{Function}{:}{}
		
		\KwData{CT scan to label, centroids}
		\KwResult{GGO label}
		
		image$\leftarrow$build\_multi\_channel\;
		\tcc{Compute distances and found the minimum}
		\ForEach{$c\in centroids $}
		{
			distances$\leftarrow\| image - c\|^2$\;
		}
		
		labels$\leftarrow\arg\min\,(distances)$\;
		
		\caption{Pseudo-code for the labeling script}\label{alg:labeling}
		
	\end{algorithm}
	

	The assignement process is performed by the \textsc{imlabeling} function, which takes care to assign the background to $0$, if the suitable parameter is passed.  The function is implemented as follows:
		
	\lstset{style=python}
	\begin{lstlisting}[language=python, caption=imlabeling, label=code:imlabeling]
		
	import numpy as np
	
	def imlabeling(image, centroids, weight = None) :


		if weight  is not None :
			distances = np.asarray([np.linalg.norm(image[weight != 0] -c, 
									axis = 1) for c in centroids])
		
			weight[weight != 0] = np.argmin(distances, axis = 0)
			return weight
		else :
			distances = np.asarray([np.linalg.norm(image -c, axis = 3) 
									for c in centroids])
			labels = np.argmin(distances, axis = 0)
			return labels
	
	
	\end{lstlisting}
	



	
	
\end{document}