\documentclass{standalone}
\begin{document}
	
	\subsection{Frameworks}
	
	In order to perform all the necessary image processing operations both involving 2D and 3D filters, to perform the color quantization and to manage the input and output medical image format, I've used mainly two libraries for image processing and computer vision. These libraries has been written in C++ but has multi language support. I've performed all the 2D image processing operations like median blurring or filter application by using \href{https://opencv.org/}{OpenCV}~\cite{OpenCV}. For the managing of medical image formats, ensuring the preservation of voxel spatial information, and for the 3D operations, I've used \href{https://simpleitk.org/}{SimpleITK}. To perform the lung extraction, I've used the pre trained Unet available \href{https://github.com/JoHof/lungmask}{here} on GitHub. In the end, to perform all the other operation on the image array, I've used \href{https://numpy.org/}{numpy}~\cite{Numpy}
	
	\subsubsection*{OpenCV} 
	
	OpenCV, acronym for Open Source Computer Vision,  is an open source computer vision and machine learning software library. OpenCV was built to provide a common infrastructure for computer vision applications and to accelerate the use of machine perception in the commercial products.
	I've used the tolls from this library to perform all the processing that involves the single image and, most important, to perform the color quantization, since the kmeans implementation offered by the library allows to cluster multi channel images in an efficient way.\\	
	This library is implemented in C++, however bindings are available for python, Java and MATLAB/OCTAVE. This library can use also hardware acceleration like Integrated Performance Primitives, and also CUDA and OpenGL based GPU interfaces are available\href{https://simpleitk.org/}{SimpleITK}.\\
	
	\subsubsection*{SimpleITK} 
	
	SimpleITK is a simplified programming interface to the algorithms and data structures of the Insight Toolkit (ITK) that support many programming languages. The library provides a simplified interface to use Insight Tool Kit(ITK) library. 
	Insight Tool Kit (ITK) is an open source library which provides an extensive suite of tools for image analysis, developed since 1999 by US National Library of Medicine of the National Institutes of Health.This library provides tool useful to works also with N-dimensional images. 
	This library provides a powerful tools for the reading and writing of the image. Since ITK, ans so SimpleITK,  consider the image like spatial object and not like arrays of values, it store also information about voxel spacing, size and origins, provided as well as the array, this makes us able to works only with the array by using numpy or OpenCV, by preserving the spatial information of the image.
	
	\subsection*{Numpy}
	
	NumPy is an open source project aiming to enable numerical computing with Python. It was created in 2005, building on the early work of the Numerical and Numarray libraries.	NumPy is developed in the open on GitHub, through the consensus of the NumPy and wider scientific Python community. For more information on our governance approach, please see our Governance Document.~\cite{Numpy}. 

	\subsubsection*{Pre-Trained UNet} 
	
	This network is a pre trained UNet which allows an Automated lung segmentation in CT under presence of severe pathologies. The whole code is written in python and it is based on torch and torchvision libraries. The repository offers 4 pre trained models for different kind of segmentation, like single lung lobe segmentation and the extraction of lung in presence of severe ILD. Each model perform the segmentation slice by slice. 
	
	The used network is a U-Net,that is a modification of the convolutional neural network architecture useful for medical and biological field, because is developed to works also with a small training dataset\\
	The network provided 3 pre-trained models : 
	\begin{itemize}
		\item \textbf{R231} : This model, trained on a dataset that cover a wide range of visual variability, performs a segmentation on individual slice and extract the right and left lung lobes including airpockets, tumor and effusion, wothout including the trachea.
		
		\item \textbf{LTRCLobes} : This model, trained on a subset of LTRC dataset, perform an individual segmentation fo lung lobes, but have limitad performances in case of severe ILD.
		
		\item \textbf{LTRCLobes\_R231} : Model which fuse the two previous one. Fills the false negative of LTCTLobes by using R231, but is computational expansive.
		
		\item \textbf{COVID231-Web}
	\end{itemize}
	
	The network was trained in order to takes the maximum flexibility with respect of the field of view, in order to enable the segmentation without a prior localization of the organs.\\
	The model is trained on $231$ CT scans collected from PACS, selected following these criteria: 
	\begin{itemize}
		\item Random Sampling ($57$ scans)
		
		\item Sampling from Image Phenotypes (71 scans)
		
		\item Manual selection of edge cases : 
			\begin{itemize}
				\item Fibrosis ($28$ scans)
				
				\item Trauma ($20$ scans)
				
				\item Other Pathology ($55$)
				
			\end{itemize}
	\end{itemize}
	
	 
\end{document}