\begin{document}
	
	\section{Implementation}
	
	In this section I will discuss in detail the pipeline implementation, focusing on how each step is achieved and implemented.
	
	The whole code is open source and available on GitHub~\cite{REP:CTLungSeg} and the pipeline installation is automatically tested on both Windows and Linux by using AppveyorCI and TravisCI.  The installation is managed by setup.py, which provides also the full list of dependencies. The code documentation was generated by using sphinx and its available online (\url{https://covid-19-ggo-segmentation.readthedocs.io/en/latest/?badge=latest}). 
	
	The pipeline provides scripts to perform lung segmentation and labeling on a single scan. It provides also an already trained set of centroids. The pipeline allows also to train other set of centroids by with the \textsc{train} script. This new set can be used instead of the default one. 
	
	Till now I've used the (silent) hypothesis that the segmentation involves only one scan. In order to automatize the segmentation on several CT scans, powershell and bash script are provided.
	
	The pipeline operates only on the image array, so no spatial informations are took into account. However, using SimpleITK, I've taken care on the preservation of these informations, in order to allows also quantitative measures.
	

\end{document} 
