\begin{document}
	
	\section{Implementation}
	
	
	I've implemented the pipeline described before using python, which is an high level object oriented programming language. To perform all necessary operations (image filtering, input/output managing, etc.) I've used three libraries: \textsc{OpenCV}~\cite{OpenCV}, \textsc{SimpleITK}~\cite{SimpleITK} and \textsc{Numpy}~\cite{Numpy}.
	
	The whole code is open source and available on GitHub~\cite{REP:CTLungSeg} and the pipeline installation is automatically tested on both Windows and Linux by using AppveyorCI and TravisCI.  The installation is managed by setup.py, which provides also the full list of dependencies. The code documentation was generated by using sphinx and its available online (\url{https://covid-19-ggo-segmentation.readthedocs.io/en/latest/?badge=latest}). 
	
	The pipeline provides $3$ scripts to perform the lung extraction, the training and the labeling. Moreover a default centroid set is provided. The provided scripts allowsto to perform lung segmentation and labeling on a single scan.  
	Till now I've used the (silent) hypothesis that the segmentation involves only one scan. In order to automatize the segmentation on several CT scans, powershell and bash script are provided.
	
	An important point is the input and output managing. The pipeline operations involves only the voxel array, but the input image formats provided also additional information like voxel spacing, image origin and direction. These information must be preserved and incorporated in the output, since are necessary to preserve the compatibility with medical image processing software. 
	To achieve this purpose I have implemented input and output method based on the implemented in \textsc{SimpleITK}. The input method allows to read an image in the medical image format and get the voxel array and the spatial information. The output one allows tosave the image in medical image format by setting also voxel array and spatial information.

	
	The construction of the multi-channel image is embedded both in training and labeling script. This step requires the application of $4$ filters on the image and to stack the together the results. For the image filtering I've used the corresponding function implemented in \textsc{OpenCV}. These functions can be applied only on a single image; so I have applied them slice by slice along the axial direction. In this way each images is processed independently from the others: No 3D structure is exploited. To stack the images I've simply used the \textsc{Numpy} \textsc{stack} method.
	Notice that the \textsc{OpenCV} functions works only with $8-$bit gray scale images. The input image must be converted from HU to GL image. This step is performed at the end of the lung extraction.
	



	
	

\end{document} 
