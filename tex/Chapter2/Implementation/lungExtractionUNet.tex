\documentclass{standalone}
\begin{document}
	\subsection{Lung Extraction}
	
	This script aims to achieve the HU registration, the lung segmentation and the bronchial removal. It takes as input the CT scan in each format supported by \textsc{SimpleITK}.  As output will return the stack after the lung extraction, saved as \textit{nifti}.  Its workflow is summarized in\ref{alg:lungExtraction}.
	
		 
	\begin{algorithm}
		
		\SetAlgoLined
		\DontPrintSemicolon
	
		\KwData{Volume (CT scan)}
		\KwResult{Volume with extracted lung}\;
		
		mask$\leftarrow$ApplyUNetVolume(Volume)\;
		
		volume$\leftarrow$(volume < -1000) =-1000\;
		volume$\leftarrow$ShiftMinimumTo1(volume)\;
		
		\tcc{Start the bronchial removal}\;
		
		eigen$\leftarrow$ maxEigenvalues(lung)\;
		bronchial\_mask$\leftarrow$ (eigen < threshold)\;
		lung\_woBronchi$\leftarrow$ (lung $\circ$ bronchial\_mask)\;
		
		\tcc{Refine the bronchial mask}\;
		
		eigen$\leftarrow$maxEigenvalues(lung\_woBronchi)\;
		bronchial\_mask$\leftarrow$(eigen < threshold)\;
		lung\_woBronchi$\leftarrow$ (lung\_woBronchi $\circ$ bronchial\_mask)\;
	
		\caption{Lung Extraction}\label{alg:lungExtraction}
		
	\end{algorithm}
	
	
	The lung mask was created by applying the pre-trained U-Net by using the suitable method from~\cite{REP:lungmask} with the pre-trained model \textsc{R231}~\cite{ART:Johannes}.

	After the creation of the lung mask, we have to achieve the registration of the HU on the whole scan. This takes care of the padding values and of the noise, that may raise at values less than $-1000\,HU$. All the values less than air one are set to $-1000$, after that all the unit are shifted to zero. Notice that set the padding values equals the air value does not raise problems, since they are removed after the application of the mask. Moreover shifting all the values to $0$ simplify the application of the mask. These operations are performed on the image tensor by using \textsc{Numpy} ndarray method. After each value is rescaled in $[0, 255]$. 
	
	\lstset{style=python}
	\begin{lstlisting}[language=python, caption=HU registering function, label=code:saf]
		
	import numpy as np
		
	def shif_and_crop(tensor) :
		
		tensor[tensor <  -1000] = - 1000
		tensor = tensor + 1000
	
		return tensor
	\end{lstlisting}

	More interesting is the estimation of the eigenvalues map. To achieve this purpose I've implemented a function based on \textsc{cv2.cornerEigenValsAndVec} in \textsc{OpenCV}. This function takes as input an $8-$bit grey scale image, so we have to rescale the GL value of the image tensor. 

	In the end the computing of the eigenvalues map is made by the following function: 
		\lstset{style=python}
	\begin{lstlisting}[language=python, caption=maximum eigenvaleus map, label=code:saf]
		
	import cv2
	import numpy as np
	from functools import patial
		
	def max_eigenvalues_map(tensor, block_size, kernel_size) :
		
		func = partial(cv2.cornerEigenValsAndVecs, 
							blockSize = block_size, 
							ksize = kernel_size)
		res = np.array(list(zip(*list(map(func, tensor)))))
		res = res.transpose(1, 0, 2, 3)
		res = res[:, :, :, :2]
		max_eigenvals = np.max(res, axis = 3)
	
		return max_eigenvals

	\end{lstlisting}

	Once the required parameters were set, the \textsc{OpenCV} function is iterate over the axial slices. The following transposition and selection of tensor elements are needed since the function returns also the eigenvectors, in which we are not interested in. 

	At the end of the whole procedure, the resulting tensor is saved into a medical image format. 
\end{document}