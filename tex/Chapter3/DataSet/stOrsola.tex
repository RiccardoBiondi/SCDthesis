\documentclass{standalone}
\begin{document}
	\subsection{Sant'Orsola}
	


	Sant'Orsola data were the ones mainly considered in this work. It consists of 83 anonymized CT scans from $83$ different patients affected by COVID-19 and $8$ scans from healthy controls. 
	Within these scans, also manual annotations were provided. These annotations were obtained with a semi-automatic approach. The built of each annotation may require several hours.
	
		The series are distributed as follows: 
	\begin{table}[h!]
		\centering
		\begin{tabular}{|c|c|}
			\hline
			\textbf{Property}   		&	\textbf{Value} \\ \hline
			Number of Scans 			& 83			   \\ 
			Distribution by sex(M/F/O)  		& 66.3/33.7/0    \\
			Distribution by age(min/median/max) & 35/60/89	\\ \hline
		\end{tabular}
	\end{table}
	
	
	For $5$ scans were provided also other semi-automatic segmentations. These segmentations were obtained by refining the initial segmentation of certified software. The building of these labels has required several days. In the end, these results are validated by $5$ experts with at least $2$ years of experience. This $ 5$ segmentation represents the gold standard used as ground truth.

	
\end{document}