\documentclass{standalone}
\begin{document}
	\subsection{Sant'Orsola}
	
	Sant'Orsola data was the ones mainly considered in this work. It consist into 83 anonimyzed CT scans from $83$ different patients affected by COVID-19 and $8$ scans from healthy controls. 
	Within these scans, also manual annotation were provided. These annotations were obtained with a semi-automatic approach. The built of each annotation may requires several hours.
	
		The series are distributed as follows: 
	\begin{table}[h!]
		\centering
		\begin{tabular}{|c|c|}
			\hline
			\textbf{Property}   		&	\textbf{Value} \\ \hline
			Number of Scans 			& 83			   \\ 
			Distribution by sex(M/F/O)  		& 66.3/33.7/0    \\
			Distribution by age(min/median/max) & 35/60/89	\\ \hline
		\end{tabular}
	\end{table}
	
	
	For $5$ scans were provided also other semi-automatic segmentations. These segmentation were obtained by refining the initial segmentation of a certified software. The building of these labels has required several days. In the end this results are validated by $5$ experts with at least $2$ years of experience. These $5$ segmentation represents the gold standard and are used as ground truth.
	
\end{document}