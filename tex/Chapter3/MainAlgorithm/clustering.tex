\documentclass{standalone}
\begin{document}

	\subsection{Clustering}

		Clustering approach is similar to classifiers one but in an unsupervised faishon, so doesn't require a training dataset.
		Clustering iteratively alternate between segmenting tha image and characterizing the  proprieties of each class. In this way we can say that clustering approach train itself by using the data available information.\\
		We can identify 3 main clustering algorithms: 
		\begin{itemize}
	
			\item \textbf{k-means clustering: } that iteratively compute a mean intensity for each class and segmentats the image by classifying each pixel in the class with the closest mean;
	
			\item \textbf{Fuzzy C-means: } this algorinthm generalize the K-means clustering in order to achieve soft- segmentation;
		
			\item \textbf{Expectation Maximization:} use the same clustering principle as k-means by assuming that the pixel follows a Gaussian mixture model. It iterates between posterior probability and compute the the Maximul Likelihood estimates for the means, covariances and mixing coefficients of the mixture model. 
	
		\end{itemize}

		This approach doesn't requires training data, but suffer to an high sensitivity to the initial parameters and do not incorporates spatial model, so it is a pixel classification technique~\cite{ART:Pham}. 

\end{document}