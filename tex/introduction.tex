\documentclass{standalone}
\begin{document}

\chapter*{Introduction}\addcontentsline{toc}{chapter}{Introduction}
\markboth{Itroduction}{Introduction}


Since the end of 2019, COVID-19 has widely spread all over the world. Up to now the gold standard for the diagnosis of this disease are the reverse transcription-polymerase chain reaction (RT-PCR) and the gene sequencing of sputum, throat swab and lower respiratory tract secretion~\cite{ART:Zhao}.

An initial prospective made by Huang and al. on chest CT scans~\cite{ART:Huang} of patients affected by COVID-19, has shown the $98\%$ of examined patients have bilateral patchy shadows or ground glass opacity (GGO) and consolidation(CS); moreover the severity, shape and involved percentage of lung were in relation with the stage of the disease~\cite{ART:Bernheim}. In the end, other study have have monitored the change on volume and shape of these features on healed patients~\cite{ART:Ai} in order to monitoring their actual recovery. In \figurename\,\ref{fig:HealthVSCovid} are compared slices of an healthy control and a patient affected by COVID-19. We can clearly see the GGO and CS regions in the lung of the second lung image from left.
	
\begin{figure}[h!]
	\centering
	\includegraphics[scale=.37]{lesionStage.png}
	\caption{Groud Glass Opacity and Consolidation on chast CT scans of COVID-19 affected patients with different severity of the disease. From left to right we can observe an increasing of the involvement of the involved lung volume.}\label{fig:HealthVSCovid}
\end{figure} 

GGO and CS are not exclusive of COVID-19, but may be also caused by pulmonary edema, bacterial infection, other viral infection or alveolar haemorrage~\cite{ART:Collins}. The combination between CT scan information and other diagnostic techniques like the RT-PCR mentioned above, may help the diagnosis, the monitoring of the course of the disease and the checking of the recovery in healed patients; moreover the study of these patterns may help to understand the infection pathogenesis, which is not well known since COVID-19 is a new disease.\\

Identification and quantification of these lesions in chest CT scans is a fundamental task. Up to now the segmentation is made in a manual or semiautomatic way, which are time consuming and subjective, since involves the interaction with trained personnel;  so an automatic and fast way for the identification of this features is desired.

This thesis work, made in collaboration with the Department of Diagnostic and Preventive Medicine of the Poloclinico Sant'Orsola - Malpighi, aims the developing of an automatic pipeline for the identification of GGO and CS in COVID-19 affected patients. The developing was based and tested on chest CT scans provided Sant'Orsola, but also public repositories~\cite{DATA:ZENODO}~\cite{DATA:MOSMED} where used as benchmark.

We start the discussion by understanding what is CT image, its physical meaning and digital representation; so a brief review on the main image segmentation techniques will be presented, describing the main features of them. More details will be given on the techniques used for the actual implementation.

The discussion will continue by describing the main pipeline characteristics and the main pipeline structure. We will see how color quantization was used to achieve the segmentation and how the digital image properties were used in order to takes into account different image features. We also discuss how a preliminary lung segmentation will help the performances of the segmentation. After that we will continue the discussion by describe in details the actual pipeline implementation, going in deep on how each step is actually achieved. Also a brief description about the used frameworks will be given.

In the end we will discuss the segmentation results. The pipeline performances were checked trough different method, like visual comparison with other segmentation techniques, quantitative comparison with a ground truth segmentation and blind evaluation by experts. Also the segmentation achieved on healthy control was considered as benchmark.

\end{document} 