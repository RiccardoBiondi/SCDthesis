\documentclass{standalone}
\begin{document}

\chapter*{Conclusions}\addcontentsline{toc}{chapter}{Conclusions}
\markboth{Conclusions}{Conclusions}

In this work I've developed a fully automated pipeline for the identification of lesions (GGO and CS) in chest CT scans of patient affected by COVID-19.
The whole pipeline achieves the segmentation by using the color quantization technique, that allows to exploit the color similarity between voxels belonging from the same tissue.  The multichannel nature of digital images has allowed also to consider other properties besides the single voxel intensity: also neighboring voxel information are takes into account.

As a preliminary step, a lung segmentation is performed by using a pre trained U-Net, followed by a bronchial removal, which aims to reduce the rate of false positives.
To achieve the segmentation, we have found the characteristic color of each tissue by performing a k-means clustering in the color space.  The actual segmentation is achieved by assigning each voxel to the cluster corresponding to the nearest color.
 

The pipeline, tested on three different datasets, has provided good segmentations for the typical lesion cases with an accurate identification of GGO and CS. The proposed algorithm has also shown some limitations due to the eventual presence of motion artifacts, caused by heartbeat and respiratory cycle; this behaviour was observed also on segmentation made on healthy controls, which have shown some misclassified points in these areas.

In the end we have developed a fully automated pipeline which achieve a good segmentation for the most typical cases. 

Color quantization has shown to be a suitable approach to face this kind of problems, and some improvements, like removal of motion artifacts are still possible.



\end{document}