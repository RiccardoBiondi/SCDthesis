\documentclass{standalone}
\begin{document}
	\chapter*{Conclusions}\addcontentsline{toc}{chapter}{Conclusions}
	\markboth{Conclusions}{Conclusions}
	
	In this thesis work I have developed a fully automated pipeline for the identification GGO and CS in CT scans of patients affected by COVID-19.
	As primary step a lung segmentation was performed using a pre trained U-Net. This is followed by a bronchial removal, which successfully reduce the number of false positives,.  
	In order to achieve the identification I have used the color quantization as medical image segmentation technique.  It allows to identify the different areas inside the lung based on color similarity. Since the lesions involves many closest voxels, I have used the color of the image to encode also neighbourhood information. This was done by assign at each channel a different image function (median blurring, gamma correction, neighbourhood standard deviation and CLAHE). To achieve the segmentation I have found the characteristic color of each tissue, performing a k-means clustering in the color space. This set can be used to segment several scans by assigning each voxel to the nearest color. 
	
	The pipeline was tested over $3$ datasets, by comparing the results with the annotation provided within. It has shown to correctly identify the GGO areas, with an high specificity, reducing the number of false positives. Even if the segmentation accuracy is not as the one of manual method, the time performances and the independence from an external expert operator make it suitable to obtain good estimations in a small amount of time. 

	In the end color quantization has shown to be a suitable approach to face this kind of problem. 
	

\end{document}