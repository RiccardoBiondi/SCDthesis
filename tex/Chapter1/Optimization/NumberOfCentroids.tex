\documentclass{standalone}
\begin{document}
	\subsection{Estimation of the Number of Clusters}
	
	The designed algorithm for the centroids estimation is the kmeans clustering that requires a prior knowledge about the number of clusters to use. This is very important since a bad chose of the number of cluster will badly affect the whole segmentation results. In order to chose the proper number of clusters, I've consider two different source of information: the anatomical knowledge about the lung and the internal variability of the lung. \\
	From anatomical knowledge about the lung, we can derive 3 clusters, corresponding to: 
	
	\begin{itemize}
		\item Lung Parenchima; 
		
		\item Ground Glass Opacity
		
		\item Alveolar structure. 
		
	\end{itemize}

	Notice that the background of the image isn't considered as a cluster since it is removed from the segmentation for the reasons explained before.\\
	In order to verify that this number of clusters is the best number of clusters, I've considered the internal cluster variability and build the elbow curve.\\
	Clustering techniques try to group the data in different clusters  in order to maximize the difference between points in different clusters and to maximize the similarity within each cluster. If the number of cluster is too low, the similarity within each cluster is low; when a proper number of cluster is chose the similarity is high and if we chose too much cluster, the similarity doesn't change too much.\\
	As a measure of the difference within clusters I've used the sum of squared error, equation\,\ref{eq:SumOfSquare}. 
	\begin{equation}\label{eq:SumOfSquare}
		SSE = \sum (x_i - c_j)^2
	\end{equation}
	In order to estimate the best number of cluster, the basic idea was to repeat the segmentation several times with different number of clusters, compute the difference within the clusters and compute the sum of square() as a measure of the difference between clusters. After that the so called \emph{elbow curve} was build, as plotted in \figurename\,\ref{fig:ElbowCurve}.
	
	\begin{figure}[h!]
		\centering
			\includegraphics[scale=.7]{PlaceHolder.png}
		\label{fig:ElbowCurve}\caption{Elbow curve}
	\end{figure}

	We can notice that the SSE decrease by increasing the number of clusters and it will be $0$ when we chose as number of cluster equal to the number of points. So we are looking at a number of clusters which provides also a small SSE. The elbow is the point from whichSSE start to decrease becaunse of the increasing of number of clusters.\\ In this case the elbbow corresponds to 3 clusters. \\
	
	Notice that this process is heuristic and will provide a benchmark about the number of cluster chosen by prior knowledge. 
	
	
	
\end{document}