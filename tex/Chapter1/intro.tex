\documentclass{standalone}
\begin{document}
	\chapter{Image Segmentation techniques}
	Image segmentation consists in the partitioning of an image into non overlapping consistent regions that are homogeneous respect to some characteristics, such as intensity or texture~\cite{ART:Pham}.
	The results of segmentation can be used to perform feature extraction, that provides fundamental information about organs or lesion volumes, cell counting, etc. If a patient perform many analysis during time, image segmentation is a useful tool to monitor the evolution of particular lesions or tumours during a therapy.
	Nowadays different non-invasive medical imaging techniques are available, such as Computed Tomography (CT), Magnetic Resonance Imaging (MRI) or X-Ray imaging, that provide a map of the subject anatomy or function. 
	
	Image segmentation plays a crucial role in many medical-imaging applications by automating or facilitate the delineation of anatomical structures and other regions of interest~\cite{ART:Pham}.  Manual segmentation is  possible, but it is time consuming and subject to operator variability making the results difficult to reproduce~\cite{INP:Withey}. Therefore an automatic or semi-automatic methods are preferable. 
	
	The major difficulty in medical image segmentation is the high variability in medical images. First and foremost, the human anatomy itself shows major modes of variation~\cite{ART:Pooja}. Furthermore the many different modalities (X-ray, CT, MRI, etc.) used to create medical images change the particular meaning of the data.
	
	In this chapter I will provide a brief introduction about medical images, focusing mainly on Computed Tomography, which is the technique used to acquire the images segmented in this work. 	
	This part is followed by a discussion on the main image segmentation techniques.
	
\end{document}