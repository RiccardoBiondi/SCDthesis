\documentclass{standalone}
\begin{document}
	\chapter{Image Segmentation techniques}
	Image segmentation consist in the partitioning of an image into non overlapping, consinstent regions that are homogeneous respect to some characteristics such as intensity or texture~\cite{ART:Pham}.
	Nowadays several non-invasive medical imaging techniques are available, such as Computed Tomography(CT), Magnetic Resonance Imaging (MRI) or X-Ray imaging, that provides a map of the subject anatomy. Image segmentation play a crucial role in many medical-imaging applications by automating or facilitate the delineation of anatomical structures and other regions of interest~\cite{ART:Pham}.  Manual segmentation is  possible, but is time consuming and subject to operator variability; making the results difficult to reproduce~\cite{INP:Withey}, so automatic or semi-automatic methods are preferable. 
	
	A major difficulty of medical image segmentation is the high variability in medical images. First and foremost, the human anatomy itself shows major modes of variation. Furthermore many different modalities (X-ray, CT, MRI, etc.) are used to create medical images~\cite{ART:Pooja}.\\
	The results of segmentation can be used to perform feature extraction, that provides fundamental information about organs or lesion volumes, cell counting, etc. If the patient perform several analysis during time, image segmentation is a useful tool to monitor the evolution of particular lesions or tumors during a therapy.
	
	In this chapter I will provide a brief introduction about medical images, focusing mainly on on Computed Tomography, which is the technique used tp acquire the images segmented in this work. 
	
	This part is followed by a discussion on the main image segmentation techniques. As before the focus will be mainly in these techniques that we have used in this work.

	
\end{document}