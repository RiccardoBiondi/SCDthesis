\documentclass{standalone}
\begin{document}

	\subsection{Clustering}

		Clustering approach is similar to classifiers one but in an unsupervised faishon, so doesn't require a training dataset.
		Clustering iteratively alternate between segmenting tha image and characterizing the  proprieties of each class. In this way we can say that clustering approach train itself by using the data available information.\\
		We can identify 3 main clustering algorithms: 
		\begin{itemize}
	
			\item \textbf{k-means clustering: } that iteratively compute a mean intensity for each class and segmentats the image by classifying each pixel in the class with the closest mean;
	
			\item \textbf{Fuzzy C-means: } this algorinthm generalize the K-means clustering in order to achieve soft- segmentation;
		
			\item \textbf{Expectation Maximization:} use the same clustering principle as k-means by assuming that the pixel follows a Gaussian mixture model. It iterates between posterior probability and compute the the Maximul Likelihood estimates for the means, covariances and mixing coefficients of the mixture model. 
	
		\end{itemize}

		This approach doesn't requires training data, but suffer to an high sensitivity to the initial parameters and do not incorporates spatial model, so it is a pixel classification technique~\cite{ART:Pham}. 
		
		
		The most used algorithm for clustering is the k-means clustering, which seek to assign each point to a particular cluster in a way that minimize the average square distance between points in the same cluster~\cite{Arthur2007}. A vector representing the mean is used to describe each cluster, so this technique is described as a centroid model~\cite{ART:Morisette}.Each point is assigned to the cluster with the nearest mean.\\
		
		Given an integer $k$ and a set of $n$ data points from $\mathbb{R}^d$, the kmeans clustering seek to find $k$ centers that minimize a potential function given by the sum of squares: 
		\begin{equation}
			\Phi = \sum_{x\in S}\min\| x - c\|^2
		\end{equation} 
		Where $S\subset \mathbb R^d$ is a set of points. In this work $\mathbb{R}^d$ is the colors space and $S$ is the space of color of each voxel.
		
		The steps of the algorithm are the following: 
		\begin{enumerate}
			\item Select the value of k as initial centroids
			\item Form k cluster by allocating every point to its most nearest centroid
			\item Recalculate the centroid for each cluster until the centroid does not change.
			
		\end{enumerate}
		
		Arthur and Vassilvitskii ~\cite{Arthur2007} have pointed that this algorithm is not accurate and can produce arbitrarily bad clusters. So they have developed a popular algorithm, the "k-means++" which improves the clustering accuracy by made an accurate choice of the initial cluster centers.\\ 
		They pointed out that the bad clustering is caused to the fact that $\frac{\Phi}{\Phi_{opt}}$ is unbounded even if the number of clusters and points are fixed, where $\Phi_{opt}$ is the potential function in the optimal centroids case. They have proposed a variant for the choosing of the centroids, instead of chose the centroids randomly, the weight the initial points according to the distance square ($D(x)^2$) from the closest center already chosen. So the final algorithm is equal to the k-means except for the initial centroids selection that is made as follows: 
		\begin{enumerate}
			\item Take one center $c_1$, chosen uniformly at random from $S$.
			
			\item  Take a new center $c_i$, choosing $x \in Si$ with probability $\frac{D(x)^2}{\sum _{x \in S} D(x)^2}$
			
			\item Repeat the step 2 till $k$ centers are choose
			
			\item Proceed like a classical k-means clustering.
			
		\end{enumerate}
		
		They have proved that this approach leads to better results in less time. For more details refer to ~\cite{Arthur2007}
\end{document}