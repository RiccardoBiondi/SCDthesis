\begin{document}
	
	\subsection{Training}
	
	This step consist in the estimation of the centroids of the color space. Is really time consung, but it is performed only once, so during the actual segmentation the corresponding script isn't run.\\
	To achieved the estimation of centroids, a kmeans clustering of the multichannel images of several CT scan from different patients is performed. 
	Since to achieve a correct estimation a huge amount of scan must be provided, this task is time consuming an computational expansive, however is performed only once and isn't directly involved in the actual segmentation, so doesn't affect the segmentation time.\\
	The achievement of this task involves two main steps : 
	\begin{enumerate}
		\item \textbf{Preparation of images} : involves the building of the multi channel images, and the registration in a common space; 
		\item \textbf{Clustering} : Actual clustering, involves also the managing of the background problem.
	\end{enumerate}

		\subsubsection*{Preparation of Images} 
	
		This step involves the preparation of images, with the building of the multi channel image that incorporates neighbouring and edges informations as well as the registration in a common space and the managing of an allocation memory problem.\\
		
		As I've said the multichannel image is build to incorporate more information during the clustering. We have found that a 4 channel image will provides good segmentation results. The 4 channel of the image are built as follows  : 
		\begin{itemize}
			\item Pure image; 
			\item Median Blurred; 
			\item Edges 
			\item Standard Deviation map. 
		\end{itemize}
	
		\begin{figure}[h]\label{fig:MultiChannel}
			\centering
				\includegraphics[scale=.55]{MultiChannel.png}
			\caption{}
		\end{figure}
	
		In \figurename\,\ref{fig:MultiChannel} I've displayed the 4 different channel of the image. The pure image will provides information about the tissue displayed in the single voxel; the median blurred allow us to consider information about the tissue surrounding each particular voxel, since lesions usually involves several group of voxels. The std map consist into the replacement of each voxel alue with the standard deviation of its neighborhood. This allow us to discriminates between the various regions and the edges, since we assume an intra-region homogeneity. Moreover this channel allow us to better discriminates bronchial regions from lesions, since bronchi have an elongated and thin structure that read in an higher std value on the neighborhood. In the end the last channel is a median blur of the edge map, which allow us to enhance the lesion regions, since presents an high number of edges.\\
		
		The first step consist into the construction of the multichannel image of for each input series, after that all the images are shuffled and divided into several subsamples. The creation of several subsamples is made since the creation of a single, huge array with several images is not always possible, since requires a huge quantity of memory to be allocated, so we have chose to divide all the images into several subsamples and cluster them independently, after that a clustering on the estimated centroids is performed.\\

		
		\subsubsection*{Clustering} 
		
		This step consist into the performing of the kmeans clustering for the centroids estimation. To perform this task I've used the OpenCV algorithm, which provides an optimized implementation of the algorithm for multi channel images. A first clustering is applied on each subsaple, resulting in a set of centroids for each one of them. On this set is applied a second clustering, which provides the actual centroids. In both of the clustering, the initial centroids set is initialized by using the kmeans ++ algorithm, which allows to improve speed and accuracy of the clustering algorithm~\cite{Arthur2007}.
		During this task we have to manage some issues. As we can see from \figurename\,\ref{fig:ClusteringHistogram} the number of voxel with $GL = 0$  is several order of magnitude higher than for other $GL$. As prior we know that these voxels belongig from background, so this cluster is over rapresented. Since kmenas cluster requres na homogenous representation for each cluster, this may raise problem during the centroids estimation. In order to overcome this issue we have simply removed this voxels from the clustering.  
		

		\begin{figure}[h]\label{fig:ClusteringHistogram}
			\centering
				\includegraphics[scale=.5]{PlaceHolder.png}
				\centering{}
		\end{figure}
		
	An other problem may be the estimation of the correct number of clusters. Kmeans clustering requires a prior knowledge on the number of clusters which is a crucial choice. In our case the anatomical knowledge about the lung may help, since we can consider one cluster for each anatomical structure. In the end we have found that 4 cluster are an optimal choice, and the considered structures are the following: 
	\begin{itemize}
		\item Lung Parenchima;
		
		\item Ground Glass Opacities;
		
		\item Bronchi;
		
		\item Eventual noise and motion artifacts. 
	\end{itemize}

	
	We don't need a cluster to represent the background, since as I' ve said before the corresponding voxel aren't takes into account during the clustering.\\
	In the end a set of centroids for each subsamples was estimated and a second clustering was performed, to found the optimal centroids. 
	This process takes a lot of time, but once we have estimated the optimal centroid set, we haven't to repeat it.\\
	
	The pseudocode of this script is reported in algorithm\,\ref{alg:training}
		
		
	\begin{algorithm}
	\label{alg:training}
	\SetAlgoLined
	\DontPrintSemicolon
	
	\SetKwFunction{FSub}{shuffle\_and\_split}
	\SetKwFunction{Fk}{kmeans\_on\_subsamples}
	\SetKwProg{Fn}{Function}{:}{}
	
	\Fn{\FSub{$images,\, number of subsamples$}}{{
	
			images$\leftarrow$shuffle(images)\;
			output$\leftarrow$split(images, number of subsamples )\;
		}
		\textbf{return} $ output $ 
	}
	\textbf{End Function}

	\Fn{\Fk{$subsamples,\, number of centroids$}}{{
			
			centroids <- []\;
			\ForEach{$ Sub \in subsamples $}
			{
				center$\leftarrow$kmeans(sub, number of centroids)\;
				centroids$\leftarrow$append(center)
				
			}
		
		}
		\textbf{return} $ centroids $ 
	}
	\textbf{End Function}
	
	\KwData{CT scans with Extracted lung}
	\KwResult{Centroid matrix}
	
	\ForEach{$scan \in input\_scans$}{
	
		read the scan\;
		sample$\leftarrow$image\_array\;
	}

	sample$\leftarrow$ build\_multichannel(sample)\;
	subsamples$\leftarrow$shuffle\_and\_split(sample, number of subsamples)\;
	centroid\_vector$\leftarrow$kmeans\_on\_subsamples(subsamples, n\_centroids)\;
	centroid$\leftarrow$kmeans\_clustering(centroid\_vector, n\_centroids)\;
	
	\caption{Pseudo-code for the training script}
	\label{Training Function}
\end{algorithm}

	
\end{document}