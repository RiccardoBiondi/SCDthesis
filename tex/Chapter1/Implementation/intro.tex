\begin{document}
	
	\section{Pipeline Implementation}
	
	In this chapter I will describe in details the actual pipeline implementation. First of all I will briefly describe the used framework fro the actual implementation. After that I will describe step by step each block of the pipeline, describing how each task is achieved.\\
	
	The whole pipeline was implemented by using python, which is an high level object oriented programming language and to perform the necessary image processing operations, the managing of input and output images, and the other operation I've mainly used OpenCV~\cite{OpenCV} and SimpleITK.\\
	Since python is an high level language, it allows an easy and fast implementation of the code, on the other hand working with optimized image processing libraries written in C++ allows to prevent the lack of performances.\\
		
	The whole code is open source and available on github~\cite{REP:CTLungSeg} and the pipeline installation is automatically tested on both Windows and Linux by using AppveyorCI and TravisCI.  The installation is managed by setup.py, which provides also the full list of dependencies. The code documentation was generated by using sphinx and its available at ... . To automatize the segmentation on multiple CT scans are provided bash and powershell script and,  even if the centroids are already estimated, a training script is provided, in order to allow the user to estimate its own set.
	
	The whole pipeline is organized into three scripts, which performs the main tasks: 
	\begin{itemize}
		\item lung\_extraction
		\item train
		\item labeling
	\end{itemize}

	The usage of SimpleITK to manage input and output file, allows the compatibility with medical image formats and the preservation of the spatial information.
\end{document} 
