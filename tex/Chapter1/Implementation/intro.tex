\begin{document}
	
	\section{Pipeline Implementation}
	
	In this chapter I will go into details about the actual pipeline implementation. First of all I will brifly describe the developing enviroment and the libraries used for the code implementation. After that I will describe how each step of pipeline is achieved.\\
	The whole pipeline was implemented by using python, which is an high level object oriented programming lenguage. The whole code is open spurce and available on github. The pipeline installation is automatically tested on both windows and linux by using AppveyorCI and TravisCI(\textbf{Insert link}).  The installation is managed by setup.py, which provides also the full list of dependencies. The code documentation was generated by using sphinx and its available at ... . Bash and powershell scripts are available to allows the segmentation on multiple patients simultaneously.
	
	The whole pipeline is organized into three scripts, which performs the main tasks of the pipeline : 
	\begin{itemize}
		\item lung\_extraction
		\item train
		\item labeling
	\end{itemize}

	For the image processing tasks, like filter application, or morphological operations, I've used itk and OpenCV libraries, which are thdescribed below. 
	The pipeline was developed in order to takes as input CT scans in medical image format like DICOM, Nifti and NRRd, and will return as output the labels in NRRD format, which can be easily read as a segmentation by medical software like 3D slicer.
\end{document}
