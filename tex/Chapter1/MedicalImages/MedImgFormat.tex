\documentclass{standalone}
\begin{document}
	\subsection{Medical Image Formats}
	
	Image file formats provide a standard way to store the information describing an image in a computer file. Moreover file format  describes how the image data are organized inside the image file and how the pixel should be interpreted by a software for the correct loading and visualization.
	
	Medical file formats can be divided in two categories: one that try to standardize the images generated by diagnostic modality(e.g. DICOM), the other that try to facilitate the post processing analysis (e.g. Nifti). Both of these type of images stores image data and metadata at the beginning of the file~\cite{ART:Larobina}. 
	
	\paragraph*{DICOM}, acronymes for Digital Imaging and COmmunications in Medicine, is not only a file format but also a network communication protocol. The added value of its adoption in terms of access, exchange, and usability of diagnostic medical images is, in general, huge.  Dicom file format establish that the pixels data cannot be separated from the description of the medical procedure which lead to the formation of the image itself. The header also contains patient informations such as name, gender, age, etc. So the header allows the image to be self descriptive. 
	DICOM is born for only 2D images, so a 3D volume is described by a series of files containing the single slices~\cite{ART:Larobina}. 
	
	\paragraph*{Nifti} primary gol is to provide coordinated and targeted service, training, and research to speed the development and enhance the utility of informatics tools related to neuroimaging. This file format uses the header to store informations about image orientation image center and origin. This avoid left-right brain hemisphere ambiguity.  Even if this file is born for neuroimaging,can be used also to store other kind of iamges like chest CT. The format is supported by many viewers and image analysis software like 3D Slicer, ImageJ, and OsiriX~\cite{ART:Larobina}. 
	
	
\end{document}