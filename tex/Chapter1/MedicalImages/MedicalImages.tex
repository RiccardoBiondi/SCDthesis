\documentclass{standalone}
\begin{document}
	\section{Medical Images}
	
	Digital images are represented by 3 dimensional tensor, where the three dimensions corresponds respectively to height, with and number of channel, where channel refers to the number of image components. For instance a Gray level image is represented with a $h\times w$ matrix, on the other hand, an RGB image is composed by three $h\times w$ matrices, each of them represent a different primary. 
	Each value of the tensor is in a range that change according to the image format. The most common are $[0, 255]$ for $8-$bit integer image, of $[0, 1]$ for float. However also other formats are available, like $16-$bit integers, widely used to represent medical images.\\

	Medical images provides a map of the subject anatomy and so are computed on a uniformly x-y-z spatial space. At each point the data is represented by a $16-$bit integer. The meaning o the data change according to the image acquisition modality(CT, MRI, PET, etc.).\\
	In computed tomography(CT), which is a technique that aims to reproduce cros-section images and the 3D anatomy of the examined subject, each data represent the capability  of the corresponding volume to attenuate an x-ray beam. In order to be able to match results from different scans the beam attenuation is measured in Housfied Units(HU) : 
	\begin{equation}\label{eq:HU}
		CT-number = k\times\frac{\mu - \mu_{H_2O}}{\mu_{H_2O}}
	\end{equation}
	
	Where $\mu$ is the linear attenuation coefficient of the tissue, $\mu_{H_20}$ is the linear attenuation coefficient of the water, took as a reference. , and $k$ is a constant which can be $1000$ or $1024$ according to manufacturer scan. The linear attenuation coefficient of the air is considered as $0$, so the corresponding CT number is $-1000$; for the bones, that have a density double than water, the CT number is $1000$. \\
	The resulting image tensor have a 16-bit depth and each voxel value is in relation with the displayed tissue.\\
		
\end{document}