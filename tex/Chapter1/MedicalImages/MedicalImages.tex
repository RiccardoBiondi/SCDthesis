\documentclass{standalone}
\begin{document}
	\section{Medical Images}
	
	A digital image is represented by 3 dimensional tensor, where the three dimensions correspond respectively to height, with and number of channel. A  Gray scale image is represented with a $h\times w$ matrix, while, an RGB image is composed by three $h\times w$ matrices, each of them represent a different primary color.
	
	Each value of the tensor is integer of floating point number belonging to a domain related to the image format. The most common are $[0, 255]$ for $8-$bit integer image, of $[0, 1]$ for float. Also other formats are available, like $16-$bit integers, widely used to represent medical images.

	Medical images allows non invasive visualization of internal organs and tissues, providing a map of the subject anatomy computed on a uniformly x-y-z spatial space. At each point the data is represented by a $16-$bit integer. The meaning o the data changes according to the image acquisition modality (CT, MRI, PET, etc.).
	
	Computed Tomography (CT) is a medical imaging technique which aims to reproduce  cross-section images and the 3D anatomy of the examined subject. Each data represents the capability  of the corresponding volume to attenuate an x-ray beam. In order to match results from different scans the beam attenuation is measured in Housfied Units(HU) : 
	\begin{equation}\label{eq:HU}
		CT-number = k\times\frac{\mu - \mu_{H_2O}}{\mu_{H_2O}}
	\end{equation}
	
	Where $\mu$ is the linear attenuation coefficient of the tissue, $\mu_{H_20}$ is the linear attenuation coefficient of the water, took as a reference, and $k$ is a constant which can be $1000$ or $1024$ according to manufacturer scan. The linear attenuation coefficient of the air is considered as $0$, so the corresponding CT number is $-1000$; for the bones, that have a density double than water, the CT number is $1000$. 
		
\end{document}